\documentclass[12pt]{article}

% ----------------------------------------- Dados do discente
% Insira os seus dados e do exercício escolhido:
\def\discente{João Lucas Rodrigues Soares}
\def\matricula{202003580}
\def\ua{08}
\def\myling{{04}} % Informe o número da linguagem selecionada.

% ------------------------------------------ Babel & Geometry
\usepackage[brazil]{babel}
\usepackage[T1]{fontenc}
\usepackage[utf8]{inputenc}
\usepackage[a4paper,top=0.5cm,bottom=1.5cm,left=1.5cm,right=1.5cm,nohead,nofoot]{geometry}
%
\usepackage{xcolor}
\usepackage{enumitem}
\usepackage{amsmath,amsfonts,amssymb,mathtools}
\usepackage[breakable]{tcolorbox}
\usepackage{booktabs}

\usepackage{tikz}
\usetikzlibrary{automata,shapes,arrows,positioning}
\tikzset{
 node distance=2.5cm,
 initial text={$M_{\mathcal{L}_\myling}$},
 double distance=1pt,
 every state/.style={semithick,fill=blue!20!white,minimum size=20pt,inner sep=0pt},
 every edge/.style={draw,->,>=stealth,auto,semithick,font=\ttfamily\small},
 accept/.style={accepting,fill=green!20!white},
 reject/.style={fill=red!20!white},
 slopea/.style={sloped,anchor=center,above},
 slopeb/.style={sloped,anchor=center,below}
}

%=-=-=-=-=-=-=-=-=-=-=-=-=-=-=-=-=-= N e w   c o m m a n d s
\newcommand{\ve}{\ensuremath{\varepsilon}}

\begin{document}
% ------------------------------------------------- Cabeçalho
 \begin{tcolorbox}[rounded corners, colback=blue!3, colframe=blue!40!black]
  \footnotesize\textbf{Universidade Federal de Goiás -- UFG}\hfill \textsc{Linguagens Formais e Autômatos -- 2022/2}\\
  \footnotesize\textbf{Instituto de Informática -- INF\hfill Prof. Humberto J. Longo} -- \scriptsize\texttt{longo@inf.ufg.br}
 \end{tcolorbox}\bigskip
%
% ------------------------------------------------- Atividade
\begin{tcolorbox}[rounded corners, colback=blue!2, colframe=blue!40!black, title=\textbf{Atividade AA-\ua}]
  Nesta tarefa deve-se ($i$) propôr um autômato finito determinístico \textbf{mínimo} $D$ que reconheça as cadeias da linguagem selecionada, a partir de $D$ construir uma gramática $G_1$ que gere as cadeias reconhecidas por $D$ e a partir de $G_1$, usando o algoritmo baseado em sistemas de equações, extrair uma expressão regular $\mathcal{R}_1$ que gere as mesmas cadeias geradas por $G_1$; ($ii$) propôr um autômato finito não-determinístico $N$ que reconheça as cadeias da linguagem selecionada e, como no item ($i$), obter a partir de $N$ uma gramática $G_2$ (não necessariamente será regular!) e, a partir desta, obter uma expressão regular $\mathcal{R}_2$. O autômato $N$ pode ser um NFA ou NFA-$\varepsilon$, com pelo menos uma transição não determinística ou uma transição $\varepsilon$. \textbf{Atenção:} NFA's criados a partir do simples acréscimo de transições $\delta(s_i,\varepsilon)=s_i$ ($\varepsilon$-laços) a um DFA não serão considerados corretos, por não permitirem uma avaliação razoável do aprendizado dos conceitos abordados nesta atividade avaliativa. (Cada aluna(o) deve consultar na descrição da atividade AA--\ua, na disciplina INF0333A da plataforma Turing, qual é a linguagem associada ao seu número de matrícula. A descrição da linguagem está disponível no arquivo ``lista de linguagens regulares'' da Seção ``Coletânea de exercícios''.)
\end{tcolorbox}\bigskip

%
% ------------------------------------ Resolução do exercício
\begin{tcolorbox}[rounded corners, colback=yellow!5, colframe=red!40!black, title={\discente\ (\matricula)}]
 \begin{itemize}[leftmargin=*]
% - - - - - - - - - - - - - - - - - - - - - - - - - - - - - - - - - - - -
  \item $\mathcal{L}_\myling = \{w\mid w$ representa um número binário e $w \geq 7$$\}$
% - - - - - - - - - - - - - - - - - - - - - - - - - - - - - - - - - - - -
  \item $ER(\mathcal{L}_\myling) = (0^+\cup 1^+)1(0^*\cup 1^*)1(0^*\cup 1^*)1(0^*\cup 1^*)$.
% - - - - - - - - - - - - - - - - - - - - - - - - - - - - - - - - - - - -
 \end{itemize}
\end{tcolorbox}\bigskip
%=========================================================================
\begin{tcolorbox}[breakable,rounded corners, colback=yellow!5, colframe=red!40!black, title={DFA mínimo que reconhece as cadeias de $\mathcal{L}_\myling$}]
 \centering
 \begin{tikzpicture}[
%    transform shape,
%    scale=0.8
 ]
  \node[state,initial]            (s0) {$S$};
  \node[state,right of=s0]        (s1) {$A$};
  \node[state,below of=s1]        (s2) {$B$};
  \node[state,right of=s1] (s3) {$C$};
  \node[state,below of=s3] (s4) {$D$};
  \node[state,right of=s4,accept] (s5) {$E$};
%
  \draw (s0) edge               node {1}   (s1)
             edge[loop above]   node {0}   (s0)
        (s1) edge               node {0}   (s2)
             edge               node {1}   (s3)
        (s2) edge               node {0,1} (s4)
        (s3) edge               node {0}   (s4)
             edge               node {1}   (s5)
        (s4) edge               node {0,1} (s5)
        (s5) edge[loop right]   node {0,1} (s5);
 \end{tikzpicture}
\end{tcolorbox}\bigskip
%=========================================================================
\begin{tcolorbox}[breakable,rounded corners, colback=yellow!5, colframe=red!40!black, title={Gramática $G_1$ que gera as cadeias de $\mathcal{L}_\myling$}]
 \begin{align*}
 G_1 &= (V,\Sigma,P,S)=(\{A,B,C,D,E,F,S\},\{0,1\},P,S),\text{ com:}\\
   P &=
   \left\{\begin{array}{l}
    S \to 0S \mid 1A,\\
    A \to 0B \mid 1C,\\
    B \to 0D \mid 1D,\\
    C \to 0D \mid 1E,\\
    D \to 0E \mid 1E,\\
    E \to 0E \mid 1E \mid \varepsilon,\\
   \end{array}\right\}.
 \end{align*}
\end{tcolorbox}\bigskip
%=========================================================================
\begin{tcolorbox}[breakable,rounded corners, colback=yellow!5, colframe=red!40!black, title={Extração de expressão regular $\mathcal{R}_1$ da gramática $G$, tal que $\mathcal{L}(\mathcal{R}_1)=\mathcal{L}(G_1)$}]
  $$
  \begin{array}{ccll}
  \toprule
  \textbf{Etapa} & \textbf{Eq.} & \textbf{Expressão} & \textbf{Ação}\\
  \midrule
    I & 1 & S = 0S \cup 1A           &\\
      & 2 & A = 0B \cup 1C           &\\
      & 3 & B = 0D \cup 1D           &\\
      & 4 & C = 0D \cup 1E           &\\
      & 5 & D = 0E \cup 1E           &\\
      & 6 & E = 0E \cup 1E \cup \varepsilon   &\\
  \midrule
   II & 1 & S = 0S \cup 1A           &\\
      & 2 & A = 0B \cup 1C           &\\
      & 3 & B = 0D \cup 1D           &I.6\texttt{ Lema de Arden}\\
      & 4 & C = 0D \cup 1E           &\\
      & 5 & D = 0E \cup 1E           &\\
      & 6 & E = (0^*\cup1)^* \cup \varepsilon   &\\     
  \midrule
  III & 1 & S = 0S \cup 1A           &\\
      & 2 & A = 0B \cup 1C           &\\
      & 3 & B = 0D \cup 1D           &II.6\to II.5\\
      & 4 & C = 0D \cup 1E           &II.6\to II.4\\
      & 5 & D = 0D\cup1((0^*\cup1)^*)\cup\varepsilon          &\\
      & 6 & E = 0((0^*\cup1)^*)\cup1((0^*\cup1)^*)\cup\varepsilon   &\\
  \midrule
   IV & 1 & S = 0S \cup 1A           &\\
      & 2 & A = 0B \cup 1C           &\\
      & 3 & B = 0D \cup 1D           &III.5\texttt{ Lema de Arden}\\
      & 4 & C = 0D \cup 1E           &\\
      & 5 & D = 0^*\cup1((0^*\cup1)^*)\cup\varepsilon          &\\
      & 6 & E = 0((0^*\cup1)^*)\cup1((0^*\cup1)^*)\cup\varepsilon   &\\
  \midrule
    V & 1 & S = 0S \cup 1A           &\\
      & 2 & A = 0B \cup 1C           &\\
      & 3 & B = 0D \cup 1D           &IV.6\texttt{ Simplificação}\\
      & 4 & C = 0D \cup 1E           &\\
      & 5 & D = 0^*\cup1((0^*\cup1)^*)\cup\varepsilon          &\\
      & 6 & E = (0\cup1)(0^*\cup1)^*\cup\varepsilon   &\\
  \midrule
    VI & 1 & S = 0S \cup 1A           &V.5\to V.3\\
      & 2 & A = 0B \cup 1C           &V.5\to V.4\\
      & 3 & B = 0(0^*\cup1((0^*\cup1)^*)\cup\varepsilon) \cup &V.6\to V.3\\
      &  &1(0^*\cup1((0^*\cup1)^*)\cup\varepsilon)           &\\
      & 4 & C = 0(0^*\cup1((0^*\cup1)^*)\cup\varepsilon) \cup &\\
      & &1((0\cup1)(0^*\cup1)^*\cup\varepsilon)           &\\
  \midrule
  VII & 1 & S = 0^*1A           &VI.3 \to VI.2\\
      & 2 & A = 0(0(0^*\cup1((0^*\cup1)^*)\cup\varepsilon) \cup &VI.4 \to VI.2\\
      &  &1(0^*\cup1((0^*\cup1)^*)\cup\varepsilon)) \cup &VI.1\texttt{ Lema de Arden}\\
      & &1(0(0^*\cup1((0^*\cup1)^*)\cup\varepsilon) \cup &\\
      & &1((0\cup1)(0^*\cup1)^*\cup\varepsilon))           &\\
  \midrule
   VIII & 1 & S = 0^*1(0(0(0^*\cup1((0^*\cup1)^*)\cup\varepsilon) \cup &\\
      &  &1(0^*\cup1((0^*\cup1)^*)\cup\varepsilon)) \cup &VII.2\to VII.1\\
      & &1(0(0^*\cup1((0^*\cup1)^*)\cup\varepsilon) \cup &\\
      & &1((0\cup1)(0^*\cup1)^*\cup\varepsilon))) &\\
  \bottomrule
  \end{array}
  $$
\end{tcolorbox}\bigskip
%=========================================================================
\begin{tcolorbox}[breakable,rounded corners, colback=yellow!5, colframe=red!40!black, title={NFA que reconhece as cadeias de $\mathcal{L}_\myling$}]
 \centering
 \begin{tikzpicture}
  [
   initial text={$N_\myling$},
   accept/.style={accepting,fill=green!20!white},
   reject/.style={fill=red!20!white},
%    node distance=2.cm,
%    transform shape,
%    scale=1.2
  ]
    \node[state,initial]            (s0) {$S$};
  \node[state,right of=s0]        (s1) {$A$};
  \node[state,below of=s1]        (s2) {$B$};
  \node[state,right of=s1] (s3) {$C$};
  \node[state,below of=s3] (s4) {$D$};
  \node[state,right of=s3,accept] (s5) {$E$};
  \node[state,below of=s5,accept] (s6) {$F$};
%
  \draw (s0) edge               node {1}   (s1)
             edge[loop above]   node {0}   (s0)
        (s1) edge               node {0}   (s2)
             edge               node {1}   (s3)
        (s2) edge               node {0,1} (s4)
        (s3) edge               node {0}   (s4)
             edge               node {1}   (s5)
        (s4) edge               node {0,1} (s6)
        (s6) edge[loop below]   node {0,1} (s6)
        (s5) edge[bend left = 20]   node {0,1} (s6)
        (s6) edge[bend left = 20]   node {0,1} (s5)
        (s5) edge[loop above]   node {0,1} (s5);
   \end{tikzpicture}
\end{tcolorbox}\bigskip
%=========================================================================
\begin{tcolorbox}[breakable,rounded corners, colback=yellow!5, colframe=red!40!black, title={Gramática $G_2$ que gera as cadeias da linguagem $\mathcal{L}_{\myling}$}]
 \begin{align*}
 G_2 &=(V,\Sigma,P,S)=(\{A,B,C,D,E,F,S\},\{0,1\},P,S), \text{ com}\\
   P &=
   \left\{\begin{array}{l}
    S \to 0S \mid 1A,\\
    A \to 0B \mid 1C,\\
    B \to 0D \mid 1D \\
    C \to 0D \mid 1E,\\
    D \to 0F \mid 1F,\\
    E \to 0E \mid 1E \mid 0F \mid 1F,\\
    F \to 0F \mid 1F \mid 0E \mid 1E\\
   \end{array}\right\}.
 \end{align*}
\end{tcolorbox}\bigskip
%%=========================================================================
\begin{tcolorbox}[breakable,rounded corners, colback=yellow!5, colframe=red!40!black, title={Extração de expressão regular $\mathcal{R}_2$ da gramática $G_2$, tal que $\mathcal{L}(\mathcal{R}_2)=\mathcal{L}(G_2)$}]
  $$
  \begin{array}{ccll}
  \toprule
  \textbf{Etapa} & \textbf{Eq.} & \textbf{Expressão} & \textbf{Ação}\\
  \midrule
    I & 1 & S = 0S \cup 1A               &\\
      & 2 & A = 0B \cup 1C               &\\
      & 3 & B = 0D \cup 1D               &\\
      & 4 & C = 0C \cup 0E               &\\
      & 5 & D = 0F \cup 1F               &\\
      & 6 & E = 0E \cup 1E \cup 0F \cup 1F              &\\
      & 7 & F = 0F \cup 1F \cup 0E \cup 1E&\\
  \midrule
   II & 1 & S = 0S \cup 1A               &\\
      & 2 & A = 0B \cup 1C               &\\
      & 3 & B = 0D \cup 1D               &\\
      & 4 & C = 0^*0E               &I.4\texttt{ Lema de Arden}\\
      & 5 & D = 0F \cup 1F               &I.6\texttt{ Lema de Arden}\\
      & 6 & E = (0^*\cup1)^* \cup 0F \cup 1F              &I.7\texttt{ Lema de Arden}\\
      & 7 & F = (0^*\cup1)^* \cup 0E \cup 1E&\\
  \midrule
  III & 1 & S = 0S \cup 1A               &\\
      & 2 & A = 0B \cup 1C               &\\
      & 3 & B = 0D \cup 1D               &\\
      & 4 & C = 0^*0E               &II.7 \to II.6\texttt{ Lema de Arden}\\
      & 5 & D = 0F \cup 1F               &II.6 \to II.7\texttt{ Lema de Arden}\\
      & 6 & E = (0^*\cup1)^* \cup 0((0^*\cup1)^* \cup (0^*\cup1)^*)&\\ & & \cup 1((0^*\cup1)^* \cup (0^*\cup1)^*)              &\\
      & 7 & F = (0^*\cup1)^* \cup 0((0^*\cup1)^* \cup (0^*\cup1)^*) &\\
      & &\cup 1((0^*\cup1)^* \cup (0^*\cup1)^*)&\\
  \midrule
   IV & 1 & S = 0S \cup 1A               &\\
      & 2 & A = 0B \cup 1C               &III.7 \to III.5\\
      & 3 & B = 0D \cup 1D               &III.6 \to III.4\\
      & 4 & C = 0^*0((0^*\cup1)^* \cup 0((0^*\cup1)^* \cup (0^*\cup1)^*)&\\ & & \cup 1((0^*\cup1)^* \cup (0^*\cup1)^*))               &\\
      & 5 & D = 0((0^*\cup1)^* \cup 0((0^*\cup1)^* \cup (0^*\cup1)^*) &\\
      & &\cup 1((0^*\cup1)^* \cup (0^*\cup1)^*)) &\\
      & &\cup 1((0^*\cup1)^* \cup 0((0^*\cup1)^* \cup (0^*\cup1)^*) &\\
      & &\cup 1((0^*\cup1)^* \cup (0^*\cup1)^*))               &\\
  \midrule
    V & 1 & S = 0^*1A               &IV.1\texttt{ Lema de Arden}\\
      & 2 & A = 0B \cup 1(0^*0((0^*\cup1)^* \cup 0((0^*\cup1)^* &IV.4 \to IV.2\\
      & &\cup (0^*\cup1)^*) \cup 1((0^*\cup1)^* \cup (0^*\cup1)^*)))               &IV.5 \to IV.3\\
      & 3 & B = 0(0((0^*\cup1)^* \cup 0((0^*\cup1)^* \cup (0^*\cup1)^*) &\\
      & &\cup 1((0^*\cup1)^* \cup (0^*\cup1)^*))\cup 1((0^*\cup1)^* &\\
      & & \cup 0((0^*\cup1)^* \cup (0^*\cup1)^*) \cup 1((0^*\cup1)^* &\\
      & & \cup (0^*\cup1)^*))) \cup 1(0((0^*\cup1)^* \cup 0((0^*\cup1)^* &\\
      & &\cup (0^*\cup1)^*)\cup 1((0^*\cup1)^* \cup (0^*\cup1)^*)) &\\
      & &\cup 1((0^*\cup1)^* \cup 0((0^*\cup1)^* \cup (0^*\cup1)^*) &\\
      & &\cup 1((0^*\cup1)^* \cup (0^*\cup1)^*)))              &\\
  \bottomrule
  \end{array}
  $$
\end{tcolorbox}
\begin{tcolorbox}[breakable,rounded corners, colback=yellow!5, colframe=red!40!black, title={Extração de expressão regular $\mathcal{R}_2$ da gramática $G_2$, tal que $\mathcal{L}(\mathcal{R}_2)=\mathcal{L}(G_2)$}]
  $$
  \begin{array}{ccll}
  \toprule
  \textbf{Etapa} & \textbf{Eq.} & \textbf{Expressão} & \textbf{Ação}\\
  \midrule
    VI & 1 & S = 0^*1A               &V.3\to V.2\\
      & 2 & A = 0(0(0((0^*\cup1)^* \cup 0((0^*\cup1)^* \cup (0^*\cup1)^*) &\\
      & &\cup 1((0^*\cup1)^* \cup (0^*\cup1)^*))\cup 1((0^*\cup1)^* &\\
      & & \cup 0((0^*\cup1)^* \cup (0^*\cup1)^*) \cup 1((0^*\cup1)^* &\\
      & & \cup (0^*\cup1)^*))) \cup 1(0((0^*\cup1)^* \cup 0((0^*\cup1)^* &\\
      & &\cup (0^*\cup1)^*)\cup 1((0^*\cup1)^* \cup (0^*\cup1)^*)) &\\
      & &\cup 1((0^*\cup1)^* \cup 0((0^*\cup1)^* \cup (0^*\cup1)^*) &\\
      & &\cup 1((0^*\cup1)^* \cup (0^*\cup1)^*)))) \cup 1(0^*0((0^*\cup1)^* \cup 0((0^*\cup1)^* &\\
      & &\cup (0^*\cup1)^*) \cup 1((0^*\cup1)^* \cup (0^*\cup1)^*)))               &\\
      \midrule
      VII & 1 & S = 0^*1(0(0(0((0^*\cup1)^* \cup 0((0^*\cup1)^* \cup (0^*\cup1)^*) &\\
      & &\cup 1((0^*\cup1)^* \cup (0^*\cup1)^*))\cup 1((0^*\cup1)^* &VI.2\to VI.1\\
      & & \cup 0((0^*\cup1)^* \cup (0^*\cup1)^*) \cup 1((0^*\cup1)^* &\\
      & & \cup (0^*\cup1)^*))) \cup 1(0((0^*\cup1)^* \cup 0((0^*\cup1)^* &\\
      & &\cup (0^*\cup1)^*)\cup 1((0^*\cup1)^* \cup (0^*\cup1)^*)) &\\
      & &\cup 1((0^*\cup1)^* \cup 0((0^*\cup1)^* \cup (0^*\cup1)^*) &\\
      & &\cup 1((0^*\cup1)^* \cup (0^*\cup1)^*)))) \cup 1(0^*0((0^*\cup1)^* \cup 0((0^*\cup1)^* &\\
      & &\cup (0^*\cup1)^*) \cup 1((0^*\cup1)^* \cup (0^*\cup1)^*))) )               &\\
  \bottomrule
  \end{array}
  $$
\end{tcolorbox}
%=========================================================================
\end{document}
%
