\documentclass[12pt]{article}

% ----------------------------------------- Dados do discente
% Insira os seus dados e do exercício escolhido:
\def\discente{João Lucas Rodrigues Soares}
\def\matricula{202003580}
\def\ua{17}
\def\up{16}
\def\myling{{24}} % Informe o número da linguagem selecionada.

% ------------------------------------------ Babel & Geometry
\usepackage[brazil]{babel}
\usepackage[T1]{fontenc}
\usepackage[utf8]{inputenc}
\usepackage[a4paper,top=1.0cm,bottom=2.0cm,left=1.5cm,right=1.5cm]{geometry}
%
\usepackage{marvosym} % \Smiley
\usepackage{xcolor}
\usepackage{enumitem}
\usepackage{amsmath,amsfonts,amssymb,mathtools}
\usepackage[breakable]{tcolorbox}
%
\usepackage{tikz}
\usetikzlibrary{automata,arrows,positioning,shapes}
\tikzset{
 node distance=3cm,
 initial text={$P_{\myling}$},
 double distance=1pt,
 every state/.style={semithick,fill=blue!20!white,minimum size=20pt,inner sep=0pt},
 every edge/.style={draw,->,>=stealth,auto,semithick,font=\ttfamily\small}
}
%
% -------------------------------------------------- New commands
%\newcommand{\tvs}{{\large\bf \textvisiblespace}}
\newcommand\vs[1][.8em]{% Visible space!
 \makebox[#1]{%
  \kern.07em
  \vrule height.5ex
  \hrulefill
  \vrule height.5ex
  \kern.07em
 }
}
\newcommand{\ve}{\ensuremath{\varepsilon}}

\begin{document}
% ------------------------------------------------- Cabeçalho
 \begin{tcolorbox}[rounded corners, colback=blue!3, colframe=blue!40!black]
  \footnotesize\textbf{Universidade Federal de Goiás -- UFG}\hfill \textsc{Linguagens Formais e Autômatos -- 2022/2}\\
  \footnotesize\textbf{Instituto de Informática -- INF\hfill Prof. Humberto J. Longo} -- \scriptsize\texttt{longo@inf.ufg.br}
 \end{tcolorbox}\bigskip
%
% ------------------------------------------------- Atividade
\begin{tcolorbox}[rounded corners, colback=blue!2, colframe=blue!40!black, title=\textbf{Atividade AA-\ua}]
 Nesta tarefa deve ser escolhida \textbf{uma} das seguintes opções:
 \begin{enumerate}
  \item Propor uma gramática livre de contexto $G_n$ que gere as cadeias da linguagem selecionada e construir, segundo um dos algoritmos apresentados nas aulas, um PDA que reconheça as cadeias da linguagem gerada pela gramática.
  \item A partir do PDA $P_n$ obtido na atividade ``AA-16 : Autômatos com pilha'', obter uma gramática que gere a linguagem aceita pelo PDA.
 \end{enumerate}
 (Cada aluna(o) deve consultar na descrição da atividade AA--\up, na disciplina INF0333A da plataforma Turing, qual é a linguagem associada ao seu número de matrícula. A descrição da linguagem está disponível no arquivo ``Lista de linguagens livres de contexto'' da Seção ``Coletânea de exercícios''.)
\end{tcolorbox}\bigskip

%
% ------------------------------------ Resolução do exercício
%=========================================================================
\begin{tcolorbox}[rounded corners, colback=yellow!5, colframe=red!40!black, title={\discente\ (\matricula)}]
\begin{itemize}%[before=\color{blue}]
  \item $\mathcal{L}_{\myling} = \{w \in \Sigma^* = \{0,1\} \mid w = 0^m1^n0^q,\ m>5,\ n>3,\ q\leq n,\ m,\ n,\ q \in \mathbb{N}\}$.
%- - - - - - - - - - - - - - - - - - - - - - - - - - - - - - - - 
  \item Gramática $G_{\myling}$ que gera as cadeias da linguagem $\mathcal{L}_{\myling}$:\\ $G_{\myling}=(V,\Sigma,P,S)=(\{S,A,B,C,D\},\{0,1\},P,S)$, com
  $
   P =
   \left\{\begin{array}{l}
    S \to AB,\\
    A \to 0A \mid 000000,\\
    B \to 1B0 \mid 1B \mid CD,\\
    C \to 1111,\\
    D \to \varepsilon \mid 0 \mid 00 \mid 000 \mid 0000
   \end{array}\right\}.
  $
%- - - - - - - - - - - - - - - - - - - - - - - - - - - - - - - - 
  \item PDA $P_{\myling}$ que reconhece as cadeias da linguagem $\mathcal{L}_{\myling}$:
  \end{itemize}
   \hspace*{-0,6cm}
   \begin{center}
   \begin{tikzpicture}[
     accept/.style={accepting,fill=green!20!white},
     reject/.style={accepting,fill=red!20!white},
     slopea/.style={sloped,anchor=center,above},
     slopeb/.style={sloped,anchor=center,below},
     transform shape,
     scale=0.9
    ]
     \node[initial,state]            (s0) {$s_0$};
  \node[state,right of=s0]        (s1) {$s_1$};
  \node[state,above of=s1]        (s2) {$s_2$};
  \node[state,above of=s2]        (s3) {$s_3$};
  \node[state,right of=s3]        (s4) {$s_4$};
  \node[state,right of=s4]        (s5) {$s_5$};
  \node[state,right of=s5]        (s6) {$s_6$};
  \node[state,right of=s6]        (s7) {$s_7$};
  \node[state,below of=s7]        (s8) {$s_8$};
  \node[state,below of=s8]        (s9) {$s_9$};
  \node[state,below of=s9]        (s10) {$s_{10}$};
  \node[state,left of=s10]        (s11) {$s_{11}$};
  \node[state,left of=s11]        (s12) {$s_{12}$};
  \node[state,accept,left of=s12] (sa) {$s_a$};
  \node[state,above of=sa]        (s13) {$s_{13}$};;
  

  \path (s0) edge node {$\ve,\ve\to \$$}               (s1)
        (s1) edge node {$0,\ve\to \ve$}         (s2)
        (s2) edge node {$0,\ve\to \ve$}       (s3)
        (s3) edge node [slopea] {$0,\ve\to \ve$}         (s4)
        (s4) edge node [slopea] {$0,\ve\to \ve$}         (s5)
        (s5) edge node [slopea] {$0,\ve\to \ve$}         (s6)
        (s6) edge node [slopea] {$0,\ve\to \ve$}          (s7)
        (s7) edge [loop above]
                  node {$0,\ve\to \ve$}                   (s7)
             edge node {$1,\ve\to x$}        (s8)
        (s8) edge node {$1,\ve\to x$}        (s9)
        (s9) edge node {$1,\ve\to x$}        (s10)
        (s10) edge node[slopea] {$1,\ve\to x$}        (s11)
        (s11) edge [loop above]
                  node {$1,\ve\to x$}                   (s11)
             edge node[slopea] {$\ve,\ve\to \ve$}        (s12)
        (s12) edge [loop below]
                  node {$0,x\to \ve$}                   (s12)
             edge [bend right=15]node[slopea] {$\ve,x\to \ve$}        (s13)
             edge node[slopea] {$\vs,\$\to \ve$}        (sa)
        (s13) edge [loop above]
                  node {$\ve,x\to \ve$}                   (s13)
             edge [bend right=15] node {$\vs,\$\to \ve$}        (sa);
        
   \end{tikzpicture}
   \end{center}
\end{tcolorbox}
%=========================================================================
\begin{tcolorbox}[rounded corners, colback=yellow!5, colframe=red!40!black, title={OPÇÂO 1: PDA construído a partir da gramática $G_{\myling}$}]
\begin{itemize}
 \item PDA $P^1_{\myling}$, construído a partir da gramática $G_{\myling}$, que reconhece as cadeias da linguagem $\mathcal{L}_{\myling}$:
   \begin{center}
   \begin{tikzpicture}[
     accept/.style={accepting,fill=green!20!white},
     slopea/.style={sloped,anchor=center,above},
     slopeb/.style={sloped,anchor=center,below},
     initial text={$P_\myling^1$},
   % transform shape,
   %  scale=0.95
    ]
     \node[initial,state]                  (s0) {$s_0$};
     \node[state,right of=s0]              (s1) {$s_1$};
     \node[state,right of=s1]              (s2) {$s_2$};
     \node[state,below of=s1,xshift=-120pt,yshift=-10pt] (s3) {$s_3$};
     \node[state,accept,right of=s2]       (s5) {$s_9$};
     \node[state,below of=s5,xshift=120pt,yshift=-10pt]  (s4) {$s_8$};
     \node[state,below of=s2,xshift=-135pt,yshift=-80pt]  (s6) {$s_4$};
     \node[state,below of=s2,xshift=85pt,yshift=-30pt]  (s7) {$s_7$};
     \node[state,below of=s2,xshift=25pt,yshift=-80pt]  (s8) {$s_6$};
     \node[state,below of=s2,xshift=-45pt,yshift=-80pt]  (s9) {$s_5$};
     
   
     \path (s0) edge node {$\ve,\ve\to \$$} (s1)
           (s1) edge node {$\ve,\ve\to S$}  (s2)
           (s2) edge [bend right=10,slopea]
                     node {$\ve,S\to B$}    (s3)
                edge node {$\ve,\$\to \ve$} (s5)
%                edge [loop above,align=center]
%                     node {$\ve,S\to B$\\[-2pt]$\ve,B\to b$\\[-2pt]
%                           $a,a\to \ve$\\[-2pt]$b,b\to \ve$}  (s2)
                edge [loop above,align=center]
                     node {$\begin{aligned}
                             0,0 &\to \ve\\[-5pt]
                             1,1 &\to \ve\\[-5pt]
                            \end{aligned}$}  (s2)
                edge [bend right=10,slopeb]
                     node {$\ve, S \to A$}  (s4)
                edge [bend right=10,slopeb]
                     node {$\ve, S \to B$}  (s6)
                edge [bend left=10,slopeb]
                     node {$\ve, S \to 0$}  (s7)
                edge [bend right=8,slopeb]
                     node {$\ve, S \to D$}  (s9)
           (s3) edge [bend right=10,slopea]
                     node {$\ve,\ve\to A$}  (s2)
           (s4) edge [bend right=10,slopea]
                     node {$\ve,\ve\to 0$}  (s2)
           (s6) edge [bend right=10,slopeb]
                     node {$\ve,\ve\to 1$}  (s2)
           (s7) edge [bend left=10,slopeb]
                     node {$\ve,\ve\to B$}  (s8)
           (s8) edge [bend right=5,slopea]
                     node {$\ve,\ve\to 1$}  (s2)
           (s9) edge [bend right=10,slopeb]
                     node {$\ve,\ve\to C$}  (s2);
   \end{tikzpicture}
   \end{center}
 \end{itemize}
\end{tcolorbox}
%=========================================================================
\end{document}
%
