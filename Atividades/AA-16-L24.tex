\documentclass[12pt]{article}

% ----------------------------------------- Dados do discente
% Insira os seus dados e do exercício escolhido:
\def\discente{João Lucas Rodrigues Soares}
\def\matricula{202003580}
\def\ua{16}
\def\up{15}
\def\myling{{24}} % Informe o número da linguagem selecionada.

% ------------------------------------------ Babel & Geometry
\usepackage[brazil]{babel}
\usepackage[T1]{fontenc}
\usepackage[utf8]{inputenc}
\usepackage[a4paper,top=1.0cm,bottom=2.0cm,left=1.5cm,right=1.5cm]{geometry}
%
\usepackage{xcolor}
\usepackage{enumitem}
\usepackage{amsmath,amsfonts,amssymb,mathtools}
\usepackage[breakable]{tcolorbox}
%
\usepackage{tikz}
\usetikzlibrary{automata,arrows,positioning,shapes}
\tikzset{
 node distance=3cm,
 initial text={$P_{\myling}$},
 double distance=1pt,
 every state/.style={semithick,fill=blue!20!white,minimum size=20pt,inner sep=0pt},
 every edge/.style={draw,->,>=stealth,auto,semithick,font=\ttfamily\small}
}
%
% -------------------------------------------------- New commands
%\newcommand{\tvs}{{\large\bf \textvisiblespace}}
\newcommand\vs[1][.8em]{% Visible space!
 \makebox[#1]{%
  \kern.07em
  \vrule height.5ex
  \hrulefill
  \vrule height.5ex
  \kern.07em
 }
}
\newcommand{\ve}{\ensuremath{\varepsilon}}

\begin{document}
% ------------------------------------------------- Cabeçalho
 \begin{tcolorbox}[rounded corners, colback=blue!3, colframe=blue!40!black]
  \footnotesize\textbf{Universidade Federal de Goiás -- UFG}\hfill \textsc{Linguagens Formais e Autômatos -- 2022/2}\\
  \footnotesize\textbf{Instituto de Informática -- INF\hfill Prof. Humberto J. Longo} -- \scriptsize\texttt{longo@inf.ufg.br}
 \end{tcolorbox}\bigskip
%
% ------------------------------------------------- Atividade
\begin{tcolorbox}[rounded corners, colback=blue!2, colframe=blue!40!black, title=\textbf{Atividade AA-\ua}]
 Nesta tarefa deve-se propor um PDA que reconheça (por estado final e pilha vazia) as cadeias da linguagem $\mathcal{L}_n$ selecionada. (Cada aluna(o) deve consultar na descrição da atividade AA--\ua, na disciplina INF0333A da plataforma Turing, qual é a linguagem associada ao seu número de matrícula. A descrição da linguagem está disponível no arquivo ``Lista de linguagens livres de contexto'' da Seção ``Coletânea de exercícios''.)
\end{tcolorbox}\bigskip

%
% ------------------------------------ Resolução do exercício
%=========================================================================
\begin{tcolorbox}[rounded corners, colback=yellow!5, colframe=red!40!black, title={\discente\ (\matricula)}]
\begin{itemize}
%- - - - - - - - - - - - - - - - - - - - - - - - - - - - - - - 
  \item $\mathcal{L}_{\myling} = \{w \in \Sigma^* = \{0,1\} \mid w = 0^m1^n0^q,\ m>5,\ n>3,\ q\leq n,\ m,\ n,\ q \in \mathbb{N}\}$.
%- - - - - - - - - - - - - - - - - - - - - - - - - - - - - - - - 
  \item PDA $P_{\myling}$ que reconhece as cadeias da linguagem $\mathcal{L}_{\myling}$:
\end{itemize}
%- - - - - - - - - - - - - - - - - - - - - - - - - - - - - - - -
%\begin{center}
\hspace*{-0,5cm}
\begin{tikzpicture}[
  accept/.style={accepting,fill=green!20!white},
  reject/.style={accepting,fill=red!20!white},
  slopea/.style={sloped,anchor=center,above},
  slopeb/.style={sloped,anchor=center,below},
  transform shape,
  scale=0.9
 ]
  \node[initial,state]            (s0) {$s_0$};
  \node[state,right of=s0]        (s1) {$s_1$};
  \node[state,above of=s1]        (s2) {$s_2$};
  \node[state,above of=s2]        (s3) {$s_3$};
  \node[state,right of=s3]        (s4) {$s_4$};
  \node[state,right of=s4]        (s5) {$s_5$};
  \node[state,right of=s5]        (s6) {$s_6$};
  \node[state,right of=s6]        (s7) {$s_7$};
  \node[state,below of=s7]        (s8) {$s_8$};
  \node[state,below of=s8]        (s9) {$s_9$};
  \node[state,below of=s9]        (s10) {$s_{10}$};
  \node[state,left of=s10]        (s11) {$s_{11}$};
  \node[state,left of=s11]        (s12) {$s_{12}$};
  \node[state,accept,left of=s12] (sa) {$s_a$};
  \node[state,above of=sa]        (s13) {$s_{13}$};;
  

  \path (s0) edge node {$\ve,\ve\to \$$}               (s1)
        (s1) edge node {$0,\ve\to \ve$}         (s2)
        (s2) edge node {$0,\ve\to \ve$}       (s3)
        (s3) edge node [slopea] {$0,\ve\to \ve$}         (s4)
        (s4) edge node [slopea] {$0,\ve\to \ve$}         (s5)
        (s5) edge node [slopea] {$0,\ve\to \ve$}         (s6)
        (s6) edge node [slopea] {$0,\ve\to \ve$}          (s7)
        (s7) edge [loop above]
                  node {$0,\ve\to \ve$}                   (s7)
             edge node {$1,\ve\to x$}        (s8)
        (s8) edge node {$1,\ve\to x$}        (s9)
        (s9) edge node {$1,\ve\to x$}        (s10)
        (s10) edge node[slopea] {$1,\ve\to x$}        (s11)
        (s11) edge [loop above]
                  node {$1,\ve\to x$}                   (s11)
             edge node[slopea] {$\ve,\ve\to \ve$}        (s12)
        (s12) edge [loop below]
                  node {$0,x\to \ve$}                   (s12)
             edge [bend right=15]node[slopea] {$\ve,x\to \ve$}        (s13)
             edge node[slopea] {$\vs,\$\to \ve$}        (sa)
        (s13) edge [loop above]
                  node {$\ve,x\to \ve$}                   (s13)
             edge [bend right=15] node {$\vs,\$\to \ve$}        (sa);
        
\end{tikzpicture}
\begin{itemize}
%- - - - - - - - - - - - - - - - - - - - - - - - - - - - - - - - 
  \item Observação importante: o estado de rejeição foi reprimido para que o PDA não fique muito extenso e dificulte a compreensão do mesmo.
\end{itemize}
%- - - - - - - - - - - - - - - - - - - - - - - - - - - - - - - -
%\end{center}
%\end{itemize}
\end{tcolorbox}
%=========================================================================
\end{document}
%
