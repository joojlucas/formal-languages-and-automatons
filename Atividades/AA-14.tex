\documentclass[12pt]{article}

% ----------------------------------------- Dados do discente
% Insira os seus dados e do exercício escolhido:
\def\discente{João Lucas Rodrigues Soares}
\def\matricula{202003580}
\def\ua{14}
\def\myling{{19}} % Informe o número da linguagem selecionada.

% ------------------------------------------ Babel & Geometry
\usepackage[brazil]{babel}
\usepackage[T1]{fontenc}
\usepackage[utf8]{inputenc}
\usepackage[a4paper,top=1.0cm,bottom=2.0cm,left=1.5cm,right=1.5cm]{geometry}
%
\usepackage{xcolor}
\usepackage{enumitem}
\usepackage{amsmath,amsfonts,amssymb,mathtools}
\usepackage[breakable]{tcolorbox}
%
\newcommand{\ve}{\ensuremath{\varepsilon}}
%\newcommand{\deriv}[1]{\stackrel{\scriptscriptstyle #1}{\Longrightarrow}}
%\newcommand{\derivG}[2]{\mathop{\Longrightarrow}\limits_{\scriptscriptstyle #2}^{\scriptscriptstyle #1}}

\begin{document}
% ------------------------------------------------- Cabeçalho
 \begin{tcolorbox}[rounded corners, colback=blue!3, colframe=blue!40!black]
  \footnotesize\textbf{Universidade Federal de Goiás -- UFG}\hfill \textsc{Linguagens Formais e Autômatos -- 2022/2}\\
  \footnotesize\textbf{Instituto de Informática -- INF\hfill Prof. Humberto J. Longo} -- \scriptsize\texttt{longo@inf.ufg.br}
 \end{tcolorbox}\bigskip
%
% ------------------------------------------------- Atividade
\begin{tcolorbox}[rounded corners, colback=blue!2, colframe=blue!40!black, title=\textbf{Atividade AA-\ua}]
 Nesta tarefa deve-se propor uma gramática livre de contexto $G_n^0$ que gere a linguagem $\mathcal{L}_n$ selecionada, ou seja, $\mathcal{L}(G)=\mathcal{L}_n$. Em seguida obtenha as gramáticas $G_n^i$, $i=1,\dots,5$, especificadas em cada um dos passos a seguir (eventualmente pode ser que $G_n^i = G_n^{i-1}$, para algum $i\in\{1,\dots,5\}$):
 \begin{enumerate}
  \item elimine recursão na variável inicial de $G_n^0$ e obtenha $G_n^1$;
  \item elimine as $\varepsilon$-regras de $G_n^1$ e obtenha $G_n^2$;
  \item elimine derivações simples de $G_n^2$ e obtenha $G_n^3$;
  \item elimine recursões à esquerda de $G_n^3$ e obtenha $G_n^4$; e
  \item elimine símbolos inúteis de $G_n^4$ e obtenha $G_n^5$.
 \end{enumerate}
 (Cada aluna(o) deve consultar na descrição da atividade AA--\ua, na disciplina INF0333A da plataforma Turing, qual é a linguagem associada ao seu número de matrícula. A especificação da linguagem está disponível no arquivo ``Lista de linguagens livres de contexto'' da Seção ``Coletânea de exercícios''.)
\end{tcolorbox}\bigskip

%
% ------------------------------------ Resolução do exercício
%=========================================================================
\begin{tcolorbox}[rounded corners, colback=yellow!5, colframe=red!40!black, title={\discente\ (\matricula)}]
\begin{itemize}
%- - - - - - - - - - - - - - - - - - - - - - - - - - - - - - - 
  \item  $\mathcal{L}_{\myling} = \{w \in \Sigma^* = \{1,0\}^* \mid w = 0^m1^m0^n,  m, n \in \mathbb{N}\}$
%- - - - - - - - - - - - - - - - - - - - - - - - - - - - - - - - 
  \item Gramática $G_{\myling}$ que gera as cadeias da linguagem $\mathcal{L}_{\myling}$:\\
  $G_{\myling}=(V,\Sigma,P,S)=(\{S,A\},\{0,1\}, P, S)$, com
    \[P=\left\{
     \begin{aligned}
      S & \to S0\mid A, \\
      A & \to 0A1\mid \ve, \\
     \end{aligned}
    \right\}\]
%- - - - - - - - - - - - - - - - - - - - - - - - - - - - - - - 
 \end{itemize}
\end{tcolorbox}\bigskip

%=========================================================================
\begin{tcolorbox}[breakable,rounded corners, colback=yellow!5, colframe=red!40!black, title={Transformações em $G_{\myling}$.}]
%-----------------------------------------------
\begin{enumerate}
  \item Eliminar recursão na variável inicial de $G_{\myling}$ e obter $G_{\myling}^1$:\\
  $G_{\myling}^1=(V,\Sigma,P,S)=(\{A,S,S_0,\},\{0,1\}, P, S_0)$, com
    \[P=\left\{
     \begin{aligned}
      S_0 & \to S,\\
      S & \to S0\mid A, \\
      A & \to 0A1\mid \ve, \\
     \end{aligned}
    \right\}\]
%- - - - - - - - - - - - - - - - - - - - - - - - - - - - - - - -
  \item Eliminar $\ve$-regras de $G_{\myling}^1$ e obter $G_{\myling}^2$:\\
  $G_{\myling}^2=(V,\Sigma,P,S_0)=(\{A,S,S_0,\},\{0,1\}, P, S_0)$, com
    \[P=\left\{
     \begin{aligned}
      S_0 & \to S \mid \ve,\\
      S & \to S0\mid A \mid 0, \\
      A & \to 0A1 \mid 01 \\
     \end{aligned}
    \right\}\]
%- - - - - - - - - - - - - - - - - - - - - - - - - - - - - - - -
  \item Eliminar derivações simples de $G_{\myling}^2$ e obter $G_{\myling}^3$:\\
  $G_{\myling}^3=(V,\Sigma,P,S_0)=(\{A,S,S_0,\},\{0,1\}, P, S_0)$, com
    \[P=\left\{
     \begin{aligned}
      S_0 & \to S0\mid 0A1 \mid 01 \mid 0 \mid \ve,\\
      S & \to S0 \mid 0A1 \mid 01 \mid 0, \\
      A & \to 0A1 \mid 01 \\
     \end{aligned}
    \right\}\]
%- - - - - - - - - - - - - - - - - - - - - - - - - - - - - - - -
  \item Eliminar recursões à esquerda de $G_{\myling}^3$ e obter $G_{\myling}^4$:\\
  $G_{\myling}^4=(V,\Sigma,P,S_0)=(\{A,S,S_0,R\},\{0,1\}, P, S_0)$, com
    \[P=\left\{
     \begin{aligned}
      S_0 & \to S0\mid 0A1 \mid 01 \mid 0 \mid \ve,\\
      S & \to 0A1 \mid 01 \mid 0 \mid 0A1R \mid 01R \mid 0R, \\
      R & \to 0R \mid 0, \\
      A & \to 0A1 \mid 01 \\
     \end{aligned}
    \right\}\]
%- - - - - - - - - - - - - - - - - - - - - - - - - - - - - - - -
  \item Eliminar símbolos inúteis de $G_{\myling}^4$ e obter $G_{\myling}^5$:\\
  $G_{\myling}^5=(V,\Sigma,P,S_0)=(\{A,S,S_0,R\},\{0,1\},P,S_0)$, com
    \[P=\left\{
     \begin{aligned}
      S_0 & \to S0\mid 0A1 \mid 01 \mid 0 \mid \ve,\\
      S & \to 0A1 \mid 01 \mid 0 \mid 0A1R \mid 01R \mid 0R, \\
      R & \to 0R \mid 0, \\
      A & \to 0A1 \mid 01 \\
     \end{aligned}
    \right\}\]
\end{enumerate}
\end{tcolorbox}
%=========================================================================
\end{document}
%
